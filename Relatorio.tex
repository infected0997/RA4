\documentclass{article}
\usepackage{booktabs}
\usepackage{geometry}
\geometry{margin=1in}

\title{Relatório de Desempenho de Algoritmos de Ordenação}
\author{Alejandro e Eduardo}
\date{}

\begin{document}

\maketitle

\section{Introdução}
Este relatório analisa o desempenho de diferentes algoritmos de ordenação: Insertion Sort, Quick Sort, Shell Sort e Counting Sort. O desempenho é avaliado em termos de tempo de execução (em nanosegundos) e número de trocas realizadas para diferentes tamanhos de entrada.

\section{Resultados}

\subsection{Desempenho dos Algoritmos}

\begin{table}[h!]
    \centering
    \begin{tabular}{@{}cccccc@{}}
        \toprule
        Tamanho & Algoritmo        & Tempo (ns)      & Trocas        \\ \midrule
        1000    & Insertion Sort   & 2,389,800      & 243,584       \\
                & Quick Sort       & 418,800        & 0             \\
                & Shell Sort       & 353,200        & 7,261         \\
                & Counting Sort    & 4,527,000      & 0             \\ \midrule
        10,000  & Insertion Sort   & 6,499,200      & 24,800,299    \\
                & Quick Sort       & 519,200        & 0             \\
                & Shell Sort       & 1,973,700      & 143,021       \\
                & Counting Sort    & 2,160,200      & 0             \\ \midrule
        100,000 & Insertion Sort   & 685,541,500    & 2,496,191,152 \\
                & Quick Sort       & 6,453,500      & 0             \\
                & Shell Sort       & 10,919,500     & 2,795,133     \\
                & Counting Sort    & 2,928,500      & 0             \\ \midrule
        500,000 & Insertion Sort   & 17,857,821,800 & 62,556,855,694\\
                & Quick Sort       & 35,919,600     & 0             \\
                & Shell Sort       & 68,413,500     & 20,711,885    \\
                & Counting Sort    & 6,503,300      & 0             \\ \midrule
        1,000,000& Insertion Sort  & 73,053,899,800 & 250,193,923,181\\
                & Quick Sort       & 75,831,200     & 0             \\
                & Shell Sort       & 154,214,200    & 50,648,787    \\
                & Counting Sort    & 8,995,500      & 0             \\ \bottomrule
    \end{tabular}
    \caption{Desempenho dos algoritmos de ordenação para diferentes tamanhos de entrada.}
    \label{tab:performance}
\end{table}

\subsection{Análise do Desempenho}
\begin{itemize}
    \item \textbf{Quick Sort}: Este algoritmo se destacou como o mais eficiente em todos os tamanhos de entrada, com tempos de execução baixos e sem trocas realizadas.
    \item \textbf{Insertion Sort}: Apresentou desempenho inferior, com tempos de execução crescentes drasticamente e um alto número de trocas, tornando-o inadequado para entradas maiores.
    \item \textbf{Shell Sort}: Ofereceu um desempenho intermediário, com tempos de execução razoáveis, mas ainda acima do Quick Sort.
    \item \textbf{Counting Sort}: Demonstrou alta eficiência em tempo, especialmente em conjuntos de dados maiores, e não realizou trocas, embora seja limitado a tipos de dados específicos.
\end{itemize}

\section{Conclusão}
Este estudo revelou diferenças significativas no desempenho dos algoritmos de ordenação analisados. O \textbf{Quick Sort} se destacou como a opção mais eficiente para entradas de grande dimensão, com tempos de execução muito inferiores e nenhuma troca, o que o torna ideal para aplicações práticas que exigem alta performance.

\end{document}
